% !TEX root = ../2021_microservices_wileytemplate.tex

\section{Introduction}

During the last ten years, cloud computing and the relatively low cost of server renting services (e.g., amazon web services, azure, google cloud) have opened the opportunity for a new style of distributed applications running on the world-wide web (instead of dedicated servers) and accessed via mobile apps (instead of desktop computers).
Cloud computing offered disruptive businesses (e.g., Netflix, Spotify, Zalando, Amazon, Uber) the necessary strategic advantage to operate at global scale servicing millions of concurrent users.

Microservices are \emph{the} architectural style that enables such ultra-scaleable applications to be deployed in the cloud.
They divide complex applications into small pieces and a fluid delivery model that delivers services on demand, thus improving performance~\cite{Larrucea2018}.
Every function in the application is implemented as a single service, deployed on its own container and accessed via an Application Programming Interface (API)~\cite{Danbettinger2019}.

The design, development, and operation of microservices are picking up more momentum in the industry.
Companies like spotify and netflix are working day-by-day on the practical implementation of microservices~\cite{Kevin2015, Meshenberg2016}.
At the same time, academic work on the topic is still at an early stage, witness mapping studies on the subject~\cite{Soldani2018, Dragoni2017, Olaf2016}.
Best practices have accumulated in a heterogeneous collection of publications, covering research as well as industrial perspectives.
As a consequence, the field is lacking a clear picture of the challenges one needs to overcome while adopting microservices and the technologies that can be used in implementing them.

In this paper, we report on a systematic literature survey on microservices, covering 81 studies from both academic and grey literature, carefully selected from a global corpus of \TOFIX{XXX} papers.
We characterise the challenges reported in the experience reports concerning the adoption of microservices and classify the technologies involved in leveraging the implementation of microservices.
We complement our classification with a discussion on possible solutions for the challenges and shortcomings for the technologies, as such establishing a research agenda for future research.


The rest of the paper is structured as follows.
In Section~\ref{sec:related-work}, we present the related work.
In Section~\ref{sec:study-design}, we describe the method and protocol followed in our systematic literature review.
In Section~\ref{sec:results}, we answer the research questions and cover the challenges and technologies used in microservices.
In Section~\ref{sec:discussion}, we discuss possible solutions for challenges mentioned in the studied literature.
In Section~\ref{sec:threats}, we cover the threats to the validity.
Finally, in Section~\ref{sec:conclusion}, we conclude the paper and outline future work.



