% !TEX root = ../final_paper.tex

\section{Related Work}\label{sec:related-work}

%\henrique{The "et al." abbreviation means more than one person (literally it means "and others"). Therefore, any reference you make to an " et al." subject is supposed to be plural, because you are referring to the authors (plural) of that paper.}

Motivated by discipline of microservices, several studies have been conducted to examine the existing literature in the field. Some of the studies on microservices are discussed in this section.
%
An interesting systematic literature review by Ghani et al.~\cite{Ghani2019} focuses on the testing challenges and quality-related aspects concerning testing. They also provide some solutions for the issues with testing. We decided to have a more broad overview of all the challenges instead of focusing on just one. Although, we did find that testing is one of the most cited challenges in our studied literature. 
%Moreover, there is a claim that their study criteria are biased during the selection of primary study and analysis

Francesco et al.~\cite{Francesco2019} motivation are on the publication trends of architecting microservices and the focus of architecting with microservices. One of their research questions is similar to our research question about the potential for industrial adoption of existing research on architecture with microservices. However, this is again focused on industrial needs whereas our study consists of a broader scope with both practitioner and academic material.
%
Pahl and Jamshidi~\cite{Claus2016} conducted a systematic mapping on microservices and it is the classification of research directions in the field and highlights the perspectives considered by researchers.

We also analyzed two survey studies.
Dragoni et al.~\cite{Dragoni2017} performed a survey on microservices. The survey gives an overview of software architecture, mostly providing the reader with references to the literature, and guiding the readers in the itinerary towards the advent of services and microservices. We focus on the adoption and implementation of microservices. We followed a systematic literature review method which is easier to reproduce.
%
Ghofrani et al ~\cite{ghofrani2018} study are similar to finding the challenges in adopting microservices but limited to survey conducted to the industrial peers. They do provide solutions from the experts to improve the aspects of the architecture. Although, it focuses on only artifacts from third parties, and not all challenges are addressed by the authors. Only the challenges faced in that company are considered by them.
