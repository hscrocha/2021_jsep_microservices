% !TEX root = ../final_paper.tex

\section{Discussion}\label{sec:discussion}

%In summary, applications built using microservices have the benefits of a strong module boundary, independent scalability, and technology diversity. Though this architecture imposes challenges, microservice adoption can be a highly beneficial experience for organizations, provided the right tools and solutions are used to address the operational and security complexities. 

%In this section, we discuss the links between the results we presented from our two Research Questions (Sections~\ref{sec:results-rq1} and~\ref{sec:results-rq2}). Therefore, we show the challenges we encountered and discuss the technologies or solutions used to address each challenge.

In this section, we describe the possible solutions for challenges that we encountered during our systematic literature review. Whenever possible we also link the technologies we presented to the appropriate challenge. Therefore, we discuss the links between the results we presented from our two research questions (Sections~\ref{sec:results-rq1} and~\ref{sec:results-rq2}), focusing on how to address the challenges, i.e., we show the challenges we encountered and discuss the solutions and technologies used to address each challenge.

\subsection{Migration}%#1

\par Before splitting or migrating the application into multiple services we need to understand its scope and architecture. Components that are used by most users should be considered first for migration. 

\par To find the candidates for microservice a method that identifies the candidates of microservices from the source code by using software clustering algorithm SArF with the relation of "program groups" and "data" can be used. The method also visualizes the extracted candidates to show the relationship between extracted candidates and the whole structure. The candidates and visualization help the developers to capture the overview of the whole system and facilitated a dialogue with customers~\cite{Kamimura2018}.

\par Database migration, mongoDB and meteor framework, both provided the necessary tools for updating the upcoming migration host. Google cloud platform (GCP) is an option for the application destination. Public cloud offered many services that would help with the goals of this optimization, it was a logical choice to migrate the application hosting server onto a Cloud Service Provider (CSP)~\cite{Tuuli2020}. Also, GCP offers its Kubernetes as a service (GKE) as part of its services. It was suitable as the target destination for the application containers~\cite{haugeland2020}. 

Migration triggers are handled by different strategies as follows~\cite{overeem2018}:
\begin{itemize}
  \item Change of model by modeler - all existing models could be changed instead of a single model.
  \item Change to the transformation environment - updated packages need to be redeployed to upgrade the instances.
  \item Change to the runtime environment - the application needs to be migrated and then Incremental migration. 
\end{itemize}

\par The migration process can be carried out by following ideas below:
\begin{itemize}
  \item Microservice identification where available software artifacts (e.g., source code, documentation, execution traces, etc.) of the monolithic application are to be analyzed to determine the corresponding microservices.
  \item Microservice packaging where the necessary transformations are to be performed on the source code of the identified~\cite{selmadji2020}.
  \item Many organizations consider a gradual process of moving to the cloud with hybrid cloud architecture~\cite{Mikail2020}.
  \item For managing common code, the study enumerated several options employed by the participants. While most participants rely on promises and challenges of Microservices~\cite{wang2020}, an exploratory study shared libraries which are the emerging sidecar technology could be a promising and elegant microservice-native solution to this problem. 
  \item Numerous options for managing application programming interface (API) are there. API gateways and client libraries, although not adopted by most of the practitioners and have a high potential to mitigate the burden related to managing API changes~\cite{wang2020}. 
  \item Options for supporting product variants are there. Each of these solutions has advantages and disadvantages, with most practitioners applying a role-based access model to support different product offerings within the same code base. 
\end{itemize}


\subsection{Performance}%#2
The increase in resource usage may cause a microservices-based application to execute slower. We can overcome this challenge by introducing additional servers. Logging in any application can be a part of the solution - we can log performance data and detect the problems. We should take advantage of logging to store performance data in a repository so that we can analyze the data at a later point in time. We can implement throttling, handle service timeouts, implement dedicated thread pools, implement circuit breakers and take advantage of asynchronous programming to boost the performance of microservice-based applications~\cite{Ghebremicael2017, Johansson2019, Zhihui2020}.

\par Apache JMeter can be used to performance-test our microservice applications. Microservice criticality factor metric (MCF) to measure the overall impact of performance scaling on a microservice from the whole application's perspective~\cite{Hou2019}. Beethoven (a platform composed of a reference architecture) could be used to measure and increase the performance of microservices~\cite{Monteiro2020}. 


\subsection{Scalability}%#3
 To scale successfully, each microservice needs to scale both individually, and as part of a larger system. Doing so requires that the dependencies of each microsystem also to scale with it. We can containerize each microservice using docker~\cite{coulson2020, Bahadori2018}.

\par Kubernetes can manage containers. Kubernetes can be configured to auto-scale based on the load. Kubernetes can identify the application instances, monitor their loads, and automatically scale up and down~\cite{McElhiney2018, khan2020}


\subsection{Testing}%#4
The best approach to solving testing challenges is adopting various testing methodologies, tools and leveraging continuous integration capabilities through automation and standard agile methodologies.
Some of the options to consider are enhanced microservice adaptive reliability testing (EMART)~\cite{Russo2020}, Apache Jmeter~\cite{Johansson2019}, testing as a service~\cite{Vanska2019}, integration testing, test doubles~\cite{Huttunen2017}, end-to-end testing, consumer-driven contract (CDC) testing,\footnote{Consumer-driven contract testing ensures that a provider is compatible with the expectations that the consumer has of it} a/b testing~\cite{Zaytev2018, Dmitrii2019}. Good quality tooling is needed like chaos monkey which will do destructive testing in the production environment to understand the resilience of system~\cite{Netflix}.


\subsection{Monitoring}%#5
We can monitor microservices by using open-source tools like prometheus with grafana APIs by creating gauges and matrices, google cloud platform (GCP) stackdriver, kubernetes monitoring, amazon cloudwatch~\cite{Kalske2017, Zhang2019, Monterio2018, Venugopal2017}.

\par Ganglia monitoring system is one of the distributed monitoring tools for high-performance computing systems~\cite{Kristiani2020}. The proposed monitoring framework includes an integrated scheduling tool, data collector, big data storage, elastic scaling manager~\cite{Zhihui2020}. For device failure detection in microservices kieker trace analysis tool is used to analyze the application and the trace tool used for visualizing are graphViz, gnuplotzunit~\cite{Saman2017}.

\par The conceptual model of a dashboard for monitoring is proposed and this model consists of components, the interaction between components, and the requirements for each component. 
A black-box approach and monitoring of the microservices through its endpoint is done~\cite{Utomo2020}.~\cite{Kevin2015} heroic tool is used for monitoring at Spotify. Atlas - cloud monitoring framework is a newly introduced tool by netflix is getting popular~\cite{Netflix}. 

\par ExplorViz uses kieker (dynamic analysis tool) to instrument software systems for a dynamical analysis of their runtime behavior~\cite{Lenga2019}. Instead of using kieker’s analysis methods, explorviz uses its own and instructs kieker to send the gathered monitoring data called records. The Analysis service creates traces from these records which are sent to the Landscape service which creates the models which are visualized in the frontend service of explorViz. The assessment confirmed that one of the central prerequisites for the effective usage of explorViz is a thorough prior knowledge of the entire monolithic architecture. The evaluation also demonstrated that finding a suitable division into self-contained systems or microservices is a highly complex process that requires experience and time.


\subsection{Security}%#6
The first step to a secure solution based on microservices is to ensure security which should be included in the design. Some fundamental tenets~\cite{Olaf2016} for all designs are: 

\begin{itemize}
\item Encrypt all communications (using https or transport layer security).
\item Authenticate all access requests.
\item Do not hard code certificates, passwords, or any form of secrets within the code.
\item Use devsecops tools designed for microservice architecture environments to scan code as it is developed.
\item Define the APIs and strictly make sure all communications comply.

\end{itemize}

In a typical microservices architecture, communication between the services can be in the same or different machines or even between different data centers. Due to this complexity in the communication, security in a microservices-based application is important to authorize access to a protected resource.
In a monolith, the facade pattern aggregates the data that is retrieved from multiple services. On the contrary, in a microservices-based application, the API Gateway is used for the same purpose. The API Gateway pattern can be used in a microservices-based application to secure access to the microservices by abstracting the underlying microservices from external clients. The other strategies that are adopted include implementation of ssl, oauth, and containerization~\cite{Zaytev2018, tenev2019, Monterio2018}. 

\par Some of the best practices in securing microservices are building security from the start, deploying security at container level, multifactor authentication, user identity and access tokens such as oauth 2.0 and openid~\cite{Gonchar2017}. Amazon~\cite{Amazon} uses amazon vpc to secure the traffic across cloud.


\subsection{Cost}%#7
Cost could be an significant factor when building microservices~\cite{Koschel2017, Netflix, Michael2018}. 
%
Adopting a microservices architecture quickly defray those costs by returning large amounts of business and technical value~\cite{McElhiney2018, leo2019}. A microservice architecture, with fewer application dependencies and simple APIs will immediately reduce the time and money spent on application maintenance. Savings on application maintenance have shown to be more than enough to cover the initial costs within a few years~\cite{Otharson2019}. 
A good set of guidelines and practices at the company level is needed. After the load was removed, the instances should be terminated automatically by aws to save cost~\cite{McElhiney2018}. 

\par Testing the off-the-shelf hardware to reduce hardware costs. Invest in the team to create white box solutions that focused on reducing costs, using a reference architecture. AWS Lambda functions can be triggered based on events ingested into amazon sqs queues, s3 buckets where aws manages the polling infrastructure on our behalf with no additional cost\cite{Ndungu2019, Zhang2019, Koschel2017}. The use of services specifically designed to deploy and scale microservices, such as aws lambda is used to reduce infrastructure costs by 70\% and Microservice implemented with play reduce up to 13\% cost~\cite{villamizar2017}.

\subsection{Complexity}%#8
Microservices architecture can be more complex than legacy applications. To handle this complexity and reduce the risks involved, we should use the right tools and technologies in place. 
Complexity can be addressed for each and every situation. If the complexity is based on integrating the legacy and internal system on the cloud services then we can use the platform as a service (PaaS) model~\cite{rosa2018} like follows application platform as a services (aPaaS), database platform services (dbPaaS), function platform services (fPaaS), business analytics platform services (baPaaS), business process management services (bpmPaaS), business rule platform services (brPaaS), enterprise horizontal portal services (Portal PaaS), communications platform services (cPaaS). When the applications are moving on-premise to software as a service (Saas), directly changing the code is not feasible because many customers share one instance of an application code. 
%
Chauvel and Solberg~\cite{chauvel2018} propose an approach to enable deep customization on multi-tenant saas using intrusive microservices.
%
\par Another possible solution is adding a service mesh to microservices that can improve visibility, monitoring, management, and security~\cite{Zaytev2018, Ndungu2019}. A service mesh allows developers to make changes without touching the application code itself. It provides the ability to mirror and monitor traffic on multiple versions of the same service, which lets developers test capabilities before deployment and determine the best way to route traffic through the system for specific types of use patterns~\cite{Premchand2018, gozneli2020}.

\par MicroBuilder is the tool used for the specification of software architecture that follows REST microservice design principles. MicroBuilder comprises MicroDSL and MicroGenerator modules. The MicroDSL module provides the MicroDSL domain-specific language used for the specification of microservice architecture~\cite{Branko2018}.


\subsection{Decomposition}%#9

In transitioning an existing monolith to a microservices, we would typically need to decompose the existing application into granular microservices~\cite{Taibi2019}. During this transitioning process, we would typically need to decompose the monolith to building more and more granular microservices to suit the business needs. Once this is accomplished, there would be more moving parts in the application~\cite{Carvalho2019}. As a result, this would lead to operational and infrastructural overheads, i.e., configuration management, security, provisioning, integration, deployment, monitoring, etc. One of the ways to reduce these complexities is by using containerization. In using containerization, provisioning, configuration, and deployment of microservices would be simplified~\cite{Zhang2019}.

\par To define the functional scope of microservices many authors propose using domain-driven design~\cite{Merson2020, neves2019, Zrzavy2020}. 
The key concepts of domain-driven design that helps in defining scopes are: 

\begin{itemize}
\item Step 1: analyze domain.
\item Step 2: define bounded context.
\item Step 3: define entities, aggregates, and services. 
\item Step 4: finally identify microservices. 
\end{itemize}

A service can have the scope of microservice. Microservice can have the scope of a bounded context. For inter-microservice interaction, we can use domain events with asynchronous messaging API calls, and service data replication.

\par Some factors must be considered when defining the size of a microservice and when we are defining a microservice granularity. Moreover, several more factors must be considered and chosen wisely to achieve a successful implementation and performance.
The most important point is balancing. Although we may consider different factors to define the level of granularity, the key purpose is to analyze all the applicable criteria and to evaluate the possible tradeoffs that will have to be made in this process. Key points to consider are as follows~\cite{Yan2020}:

\begin{itemize}
\item Get a clear picture of the solution architecture. 
\item Functional decomposition patterns must be applied, this helps in defining services and splitting them.
\item Separate reusable activities into reusable services.
To achieve the above, the key technologies used were service registration and discovery, remote service calling, circuit-breaker mechanism, service link tracking, and annotation interfaces.
\end{itemize}


\subsection{Integrating}%#10
Integration of APIs, services, data, and systems has been a challenging yet essential requirement in the context of enterprise software application development. Before integrating all of these disparate applications in point-to-point style was done, which was later replaced by the enterprise service bus (ESB) style, alongside the service oriented architecture (SOA).
Integration platform services (iPaaS) provides a platform in the cloud to support the application, data, and system-to-system integrations, using a mix of cloud services, mobile apps, on-premises systems, and internet of things integrations. 

\par A message-Oriented middleware services (momPaaS) was proposed to support the communication and integration between the different microservices implemented in the respective system, thus supporting the message exchange with different protocols.
Enterprise horizontal portal services (Portal PaaS) can be used to provide a B2B portal that is integrated with the microservices layer.
Integration platform services (iPaaS) was recommended for situations when the integration and exchange of information between cloud applications and on-premise and legacy applications are required ~\cite{rosa2018}.

\par Architecture style makes it possible to achieve fine-grained incremental migration. The integration of this architecture style manifests itself in two ways. First of all, the runtime environment should be able to host distributed microservices. The benefits of this are an improved upgradability, scalability, resilience, and resource sharing.
Second, the transformation environment should consist of independent microservices. This enables the incremental transformation of the model through the pattern incrementality by traceability~\cite{liu2018, overeem2018}.

\par Continuous integration and continuous delivery both go hand in hand with microservices. Without these two practices, it becomes very hard to handle multiple services, their deployments, and validating the actions of the service~\cite{Zhang2019, Kalske2017}. 


\subsection{Communication}%#11

Two types of communication are taken into consideration. One is the communication between teams and the other is communication between services.

\par Often, enhancement or corrective maintenance issues (e.g. interface model changes, design decision changes) must be communicated between multiple teams. 'Multi-project coding issues' as a solution approach for communicating issues concerning multiple projects/services and teams in a qualitative way. A multi-project coding issue is a coding issue that can concern more than one project at the same time. They can link other coding issues that could concern other projects/services~\cite{Speth2019}.

\par Unlike the internals of a monolith, microservices communicate over a network. In some circumstances, this can be seen as a security concern. Istio solves this problem by automatically encrypting the traffic between microservices~\cite{Zhang2019}.
Modules could be communicated using coap which has minimal overhead\cite{liu2018}. Apache kafka is one of the right solutions for communication between microservices and it is been used at ebay~\cite{Gonchar2017}.



\subsection{Logging}%#12

One of the main criteria for a mature microservice-based development process is the robustness of the logging. It is better to be set up as early as the project starts~\cite{wang2020}.
Detecting failures quickly for automatically restoring services is critical and is typically achieved using logging and monitoring tools, such as grafana, kibana, logstash, and elastic stack~\cite{KalskeM2017}.
 
\par Grafana has released loki, a solution meant to complement the main tool to better parse, visualize and analyze logging~\cite{Kalske2017, Zhang2019}. It is hard to identify issues between microservices when services are dependent on each other and they have a cyclic dependency. Correlation id can be passed by the client in the header to REST APIs to track all the relevant logs across all the pods/docker containers on all clusters. Zap app is been used at uber for logging~\cite{Matt2016}.


\subsection{Technical challenges}%#13

\par Microservice architecture can be very useful but it also comes with technical challenges that have to be solved to get complete benefit from them. These challenges could be handling of distributed transactions, communication between microservices, separation of concerns in microservices etc.~\cite{Kalske2017,KalskeM2017}. We provide some ideas on how to overcome technical challenges below as follows.

\begin{itemize}
\item For data consistency and distributed transactions- a saga patterns \footnote{Long lived transactions (LLTs) hold on to database resources for relatively long periods of
time, significantly delaying the termination of shorter and more common transactions to alleviate these problems the notion called saga is proposed. A LLT is a saga if it can be written as a sequence of transactions that can be interleaved with other transactions. The database management system guarantees that either all the transactions m a saga are successfully completed or compensating transactions are run to amend a partial execution} could be implemented and avoid a two-phase commit ~\cite{neves2019}.
\item Testing complexity - integration tests and following domain driven design concepts.
\item Setup and execution of the initial prototype - When the microservices architecture is being developed from scratch the team should try to use the monolith-first approach and then migrate.
\item Creating uniformity across multiple services - static code analysis tools can be used.
\item Distributed monitoring - the concept of Log aggregation should be used.
\item Decomposition of the pre-existing system with the proper granularity and low coupling - most successful microservices migrations identified used Domain Driven Design to deal with this issue.
\item Database challenges - usage of circuit breaker. 
\item Communicational challenges - REST API.
\item Testing challenges - unit testing, integration testing and end-to-end testing.
\item Observability challenges - zookeeper, eureka server, docker.
\item Organizational challenges - experimenting should not be restricted~\cite{Kalske2017, KalskeM2017}.

For some of the application system it is good to mainly focus on technology choice. Strong knowledge of devops and cloud architecture patterns is requirement~\cite{Falatiuk2019}.
\end{itemize}


\par An orchestration plan describes components, their dependencies, and their lifecycle in a layered plan. A paas then enacts the workflows from the plan through agents. Paas can support the deployment of applications from containers~\cite{Sharaf2019}. In paas, there is a need to define, deploy and operate cross-platform capable cloud services using light-weight virtualization, for which containers are a solution. There is also a need to transfer cloud deployments between cloud providers, which requires lightweight virtualized clusters for container orchestration. Some paas are lightweight virtualization solutions in this sense. 

Specific or generic solutions for paas platforms include 
\begin{itemize}
\item Linux: Docker, lxc linux containers, openvz, and others for variants such as bsd, hp-ux, and solaris. 
\item Windows: sandboxie 
\item Cloud paas: Warden/Garden (in cloud foundry), lxc (in openshift). 
\end{itemize}

Kubernetes is a more comprehensive solution that would tackle orchestration of complex application stacks that could involve docker orchestration based on the topology-based service orchestration standard tosca, which is for instance supported by cloudify paas \cite{Bahadori2018, Falatiuk2019, Venugopal2017 ,leo2019, Kalske2017, Zaytev2018, Kristiani2020, khan2020}. 


\subsection{Technical Debt}%#14
The technical debts can cover a lot of areas but we address only the technical debt in code quality, architectural debt. There are two solutions, the first one is clean up, and the second is prevention.
 
\par For cleaning up, we can perform parallel rewrite, partial rewrite, and phaseout; for prevention: code reviews, acceptance criteria, workflow, and tests~\cite{Tuuli2020}. domain-driven design is one of the solutions to reduce technical debt. 
It enables us to organize a testable layer of code that is similar to our problem domain and is responsible for all the business logic of our app~\cite{Zrzavy2020}. If the microservices configuration is done badly then the common way to fix this is to use configuration servers like Spring cloud config which handles the configuration by itself. Service discovery tools like eureka and consul could help too. When the debt is due to libraries used by some developers then the solution is to use multiple different libraries that have well-defined boundaries~\cite{Uber}. 


\subsection{Documentation}%#15
Documentation of microservices can be time-consuming and challenging that is why we need to follow some of the solutions like comprehensive documentation which should include description, architecture diagram, endpoint, dependencies, run books, contact information, onboarding guide, etc. The other solution is to update documentation as a part of the development cycle. Using a central location for all microservice documentation~\cite{selmadji2020}.

\par There are couple of challenges such as content, assignment, tooling, and business-related when documenting the microservice-based information techology landscape from an enterprise architecture (EA) model maintenance perspective. To document the EA model, Kleehaus~\cite{kleehaus2019} suggests a solution of automatically documenting the runtime data from the monitoring tool. However, the tool used for automated documentation is not yet described in his publication and he mentions to publish it in the future works.



\subsection{IoT Challenges}%#16
Containers are particularly useful to enable portability and heterogeneity for high-degree automation in testing, deployment, and operations. Apache kafka and spark for application resiliency.
The Akka, and Play framework are useful in building individual services, and leveraging both http and kafka for inter-service communication~\cite{khan2017}.

\par The combination of edge and cloud computing is going to make the internet of things (IoT) rapid, light, and more reliable. IoT and cloud-edge computing are distinct disciples that have evolved separately over time. However, they are increasingly becoming interdependent, and are what the future holds. A crucial aspect is how to design a compound of cloud and edge computing architectures, and implement IoT effectively.
An implementation of container-based virtualization which constructs kubernetes minion (nodes) in the socker container service independently for each service on the edge side can address IoT challenges. Moreover, docker and kubernetes when combined together provide the best solutions by effectively organizing containerized applications~\cite{Kristiani2020}.



\subsection{Power management}%#17
There are couple of way to manage the power consumption which are described in the auto, native and transparent power management framework (ANT-man) solution. 
First, an auto power budgeting scheme for reducing the power coordination latency at the datacenter level. It can proactively determine the power budget tailored to each microservice. Second, ANT-Man proposes a native and transparent power control scheme to overcome the power configuration latency for each microservice. It enables super-fast power budget enforcement with nanosecond-scale performance scaling~\cite{Hou2020}.

\par To unleash the performance potential of the cloud in the microservice era, we model it using a bipartite graph and a metric called microservice criticality factor (MCF) to measure the overall impact of performance scaling on a microservice from the whole application's perspective proposed. A novel system framework called servicefridge that leverages MCF to jointly orchestrate software containers and control hardware power demand is also proposed. servicefridge allows the data center to reduce its dynamic power by 25\% with slight performance loss. It improves the mean response time by 25.2\% and improves the 90th tail latency by 18\% compared with existing schemes~\cite{Hou2019}.


\subsection{Load balancing}%#18
Load balancers are used to distribute incoming traffic to the network by efficiently distributing across multiple servers. By doing this reliability and high availability are maintained. It makes it easier to use in adding and removing servers in the network as per demand. 

\par The API gateway can also handle authorization and load balancing for the microservices. The benefits of using an API gateway was reported to be extensibility and backward compatibility because of the additional level of freedom the customized API can provide. The reported downsides~\cite{Taibi2019} were scalability issues and increased complexity in the architecture. Because graphql enables clients to define their customized queries, a single graphql endpoint can be used by multiple clients, thus graphql suits well to be used in the API gateway pattern~\cite{Ville2019}.

\par To resolve challenges like load balancing, user acceptance, routing, and microservice auto-discovery. A redundant program code that covers the same functionality needs to be written at different layers of the software architecture. This often leads to mistakes, as a developer introduces unintentional errors to the repetitive code constructs. Moreover, the configuration of microservice architecture is not a trivial task and it requires intricate and redundant work to be performed. Therefore, to mitigate and speed up such a process, it could be beneficial for a developer to have a language with a concise set of concepts that are specific to the domain of REST microservice architecture development. Such a language should allow developers to have a single specification of a microservice without writing any boilerplate or redundant code.
microbuilder tool frameworks provide a set of tools such as (i) zuul, for user request routing and filtering (ii) eureka, for microservice auto-discovery and registry, (iii) ribbon, for resilient and intelligent inter-process communication and load balancing, and (iv) hystrix, for isolated latency and fault tolerance among microservices. These tools are utilized to resolve some of the challenges caused by a large number of microservices per application. The remaining challenges concerning redundant coding and not so trivial configuration of microservice architectures, using these tools can be resolved by using microBuilder too~\cite{Branko2018}.





