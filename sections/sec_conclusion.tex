% !TEX root = ../2021_microservices_wileytemplate.tex

\section{Conclusion}\label{sec:conclusion}

In this paper, we reported the results of our systematic literature review in microservices. We collected and read a total of 81 references from both scientific and grey literature sources we deemed relevant to answer our research questions. Our work can be used as groundwork and complementary to the existing literature reviews to guide researchers to open issues and challenges in microservices and offer an overview of solutions to consider.

The first research question addresses the challenges in adopting microservices. Among the collected publications, we identified \challengecount different challenges. The most mentioned challenges were migration (14 references), performance (11 references), scalability (10 references), and testing (10 references). The least mentioned were power management (2 references) and load balancing (2 references). 

The second question addresses the technologies used in implementing microservices. We distinguished a total of \techcount reported technologies which we grouped into the following categories: container, programming languages, other technologies, communication, framework, and platform as a service. The most mentioned technology categories were Container (14 references), programming languages (13 references). Other technologies also have many references to be top-ranked (13 references), however, it is a miscellaneous category with unrelated technologies. When we look at the ungrouped technologies, the most cited are Docker (12 references), Kubernetes (9 references), and REST (8 references).

We also discussed the possible solutions for each of the challenges. The solutions may vary based upon the requirement and necessity of adoption of microservices. According to the gathered solution, a necessary aspect to remember is to analyze the entire system before migrating to microservices.

Future work includes (i) adding more grey literature and valuable resources; (ii) conducting a survey with the industrial peers; and (iii) providing some broader scope to the collected materials about the aspects and architecture of microservices.







