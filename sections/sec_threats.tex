% !TEX root = ../2021_microservices_wileytemplate.tex

\section{Threats to validity}\label{sec:threats}
In this section we acknowledge the threats to validity in the process we use in our systematic
literature review. We discuss each threat and its mitigation strategy based on the guidelines presented by Kitchenham and Charters.\cite{Kitchenham2007}


\subsection{Construct Validity} 
Construct validity concentrates on the sufficiency of the study's design to address the research questions. Many trials and discussions were carried out to define a search string and mitigate any subjectiveness in our study. The search string that resulted in the most relative number of results was selected. To mitigate threats related to the study selection strategy, the strategy was based on software engineering systematic review guidelines.\cite{kitchenham2015}
%\keerthana{i cited kitchenham 2015 because it has construct validity in page 25,35}

 
\subsection{Internal Validity} 
Internal validity is concerned with the conduct of the study. To mitigate internal validity threats during the study selection process, we followed the guidelines presented by Kitchenham and Charters\cite{Kitchenham2007} to construct the search strategy and prevent any systematic error.
The procedure and its implementation were discussed by the first and second authors to mitigate any subjectiveness in our study. The research questions we defined helped in selecting relevant studies. However, the chosen inclusion and exclusion criteria might have led to missing contributions that could have inspired the microservices field. While extracting videos for inclusion in the study results, we tried to include videos as much as we could using the search keywords the author might have mentioned during his presentation. We mitigated this threat by having two persons (the first and second authors) going over the selection process separately. Moreover, if one person involved voted to include a reference, it was included. Therefore, any subjective bias on the inclusion or exclusion of references was lessened. 

\subsection{External Validity}
External validity is concerned with the generalizability of a study’s
findings.\cite{Kitchenham2007} Since our primary studies are obtained from a large extent of online sources, our results and observations may be only partly applicable to the broad area of practices and general disciplines of microservices. Hence to mitigate this threat we performed multiple iterations of backward snowballing to expand the search scope.
%
Though the aim was to
cover a representative body of implementation and challenges of microservices prioritization literature, the findings may not have prioritized specific challenges or implementation technologies outside of the primary studies. 
Moreover, there is a risk of having missed relevant industrial studies, because concepts related to those included in our search strings are differently named in such studies %(e.g., a study discussing architecture of microservices may not employ the terms “challenges” but rather some synonyms). 
To mitigate this threat to validity, we defined and followed a protocol where several search strings were considered, until we settled on the one with the most appropriate results to answer our research questions.



\subsection{Conclusion Validity} 
Threats to conclusion validity are related to issues that affect the ability to draw the correct conclusions from the study.\cite{Kitchenham2007} From the reviewers’ perspective, a potential threat to conclusion validity is the reliability of the data extraction categories from the selected sources. The technologies and challenges collected during the systematic literature review might be limited to the collected evidence. We are aware that the microservices challenges could be addressed by a wide range of solutions but we have answered according to the solutions we could find using the studied references. Although we did not explicitly address this threat, we claim it was mitigated due to the heterogeneity of the references we used. Additionally, to ensure validity multiple sources of data were analyzed, i.e., articles, blogs, news, and videos. %Furthermore, in the event of a disagreement between the one reviewer, the second reviewer acted as an arbitrator to ensure the agreement was reached.

